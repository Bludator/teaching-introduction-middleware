\part{Sockets: The Hard Way}


\section{Berkeley Socket Interface}


\begin{frame}{Interface Overview}
    \begin{block}{Socket}
        An abstraction representing a (network) communication channel. \\
        Both stream oriented and message oriented channels. \\
        Spectrum of supported protocols. \\
    \end{block}

    \begin{block}{Stream Oriented Channel}
        Socket on \emph{client side} initiates outgoing connections. \\
        Socket on \emph{server side} waits for incoming connections. \\
        Data flows in both directions after connection established. \\
    \end{block}

    \begin{block}{Message Oriented Channel}
        No connection established. \\
        Sender and receiver roles symmetrical. \\
    \end{block}
\end{frame}


\begin{frame}[fragile]{Stream Oriented Channel}
    \begin{block}{Client Side Pseudocode}
\begin{lstlisting}[style=mini]
socket = CreateSocket (comms_domain, socket_type);
ConnectToServer (socket, server_address);
... Write (socket, data);
... Read (socket, data);
Shutdown (socket);
Close (socket);
\end{lstlisting}
    \end{block}
    \begin{block}{Server Side Pseudocode}
\begin{lstlisting}[style=mini]
server_socket = CreateSocket (comms_domain, socket_type);
BindToLocalAddress (socket, address);
PermitListeningOnSocket (socket, backlog);
client_socket, client_address = AcceptIncomingConnection (socket);
... Write (client_socket, data);
... Read (client_socket, data);
Shutdown (client_socket);
Close (client_socket);
\end{lstlisting}
    \end{block}
\end{frame}


\begin{frame}[fragile]{Examples To Play With ...}
\begin{lstlisting}[language=bash,style=mini]
> git clone http://github.com/d-iii-s/teaching-introduction-middleware.git
\end{lstlisting}
    \begin{block}{C}
\begin{lstlisting}[language=bash,style=mini]
> cd teaching-introduction-middleware/src/sockets-basic-server/c
> cat README.md
\end{lstlisting}
    \end{block}
    \begin{block}{Java}
\begin{lstlisting}[language=bash,style=mini]
> cd teaching-introduction-middleware/src/sockets-basic-server/java
> cat README.md
\end{lstlisting}
    \end{block}
    \begin{block}{Python}
\begin{lstlisting}[language=bash,style=mini]
> cd teaching-introduction-middleware/src/sockets-basic-server/python
> cat README.md
\end{lstlisting}
    \end{block}
\end{frame}


\section{Assignment Part I}


\begin{frame}{Assignment}
    \begin{block}{Server}
        Implement a server that will:
        \begin{itemize}
            \item Listen for incoming connections.
            \item Provide information on current time to connected clients.
        \end{itemize}
    \end{block}

    \bigskip

    \begin{block}{Client}
        Implement a client that will:
        \begin{itemize}
            \item Connect to the server described above.
            \item Query information on current time.
            \item Wrap all this in a local function.
            \item Print the time.
        \end{itemize}
    \end{block}
\end{frame}


\begin{frame}[fragile]{C Local Function}
\begin{lstlisting}[language=c,style=mini]
/**
 * Return server time in standard structure.
 * \param result Caller allocated structure to fill.
 * \return Zero for success, non zero error code otherwise.
 */
int server_time (struct tm *result);

struct tm {
    int tm_sec;    // Seconds (0-60)
    int tm_min;    // Minutes (0-59)
    int tm_hour;   // Hours (0-23)
    int tm_mday;   // Day of the month (1-31)
    int tm_mon;    // Month (0-11)
    int tm_year;   // Year - 1900
    int tm_wday;   // Day of the week (0-6, Sunday = 0)
    int tm_yday;   // Day in the year (0-365, 1 Jan = 0)
    int tm_isdst;  // Daylight saving time
};
\end{lstlisting}
    \hfill ... man \lstinline{localtime}
\end{frame}


\begin{frame}[fragile]{Java Local Function}
\begin{lstlisting}[language=java,style=mini]
/**
 * Access server time in standard structure.
 */
public interface ServerTime {
    int getSecond ();           // Gets the second-of-minute field.
    int getMinute ();           // Gets the minute-of-hour field.
    int getHour ();             // Gets the hour-of-day field.
    int getDayOfMonth ();       // Gets the day-of-month field.
    Month getMonth ();          // Gets the month-of-year field.
    int getYear ();             // Gets the year field.
    DayOfWeek getDayOfWeek ();  // Gets the day-of-week field.
    int getDayOfYear ();        // Gets the day-of-year field.
}
\end{lstlisting}
    \hfill ... javadoc \lstinline{LocalDateTime}
\end{frame}


\begin{frame}[fragile]{Python Local Function}
\begin{lstlisting}[language=python,style=mini]
def server_time ():
    """Returns server time in datetime.datetime class."""
    ...

# Instance attributes (read-only):
#
#     datetime.year
#         Between MINYEAR and MAXYEAR inclusive.
#     datetime.month
#         Between 1 and 12 inclusive.
#     datetime.day
#         Between 1 and the number of days in the given month of the given year.
#     datetime.hour
#         In range(24).
#     datetime.minute
#         In range(60).
#     datetime.second
#         In range(60).
\end{lstlisting}
    \hfill ... \lstinline{help (datetime.datetime)}
\end{frame}


\section{Marshalling Implementation}


\begin{frame}[fragile]{C Marshalling}
    \begin{block}{Textual Stream ?}
\begin{lstlisting}[language=c,style=mini]
int sprintf (char *str, const char *format, ...);
int sscanf (const char *str, const char *format, ...);
\end{lstlisting}
    \end{block}
    \begin{block}{Network Order Binary Stream ?}
\begin{lstlisting}[language=c,style=mini]
uint32_t htonl (uint32_t hostlong);
uint16_t htons (uint16_t hostshort);
uint32_t ntohl (uint32_t netlong);
uint16_t ntohs (uint16_t netshort);
\end{lstlisting}
    \end{block}
    \begin{block}{Native Order Binary Stream ?}
\begin{lstlisting}[language=c,style=mini]
char buffer [1024];
int *address = (int *) &buffer [16];
*address = 1234;
\end{lstlisting}
    \end{block}
\end{frame}


\begin{frame}[fragile]{Java Marshalling}
    \begin{block}{Serialized Stream ?}
\begin{lstlisting}[language=java,style=mini]
output_stream = socket.getOutputStream ();
object_stream = new ObjectOutputStream (output_stream);
object_stream.writeInt (1234);
object_stream.writeObject (...);
\end{lstlisting}
    \end{block}
    \begin{block}{Textual Stream ?}
\begin{lstlisting}[language=java,style=mini]
PrintWriter writer = new PrintWriter (output_stream, true);
writer.println ("...");
\end{lstlisting}
    \end{block}
    \begin{block}{Byte Stream ?}
\begin{lstlisting}[language=java,style=mini]
ByteBuffer buffer = ByteBuffer.allocate (4);
buffer.putInt (1234);
output_stream.write (buffer.array ());
\end{lstlisting}
    \end{block}
\end{frame}


\begin{frame}[fragile]{Python Marshalling}
    \begin{block}{Pickled Stream ?}
\begin{lstlisting}[language=python,style=mini]
import pickle
with socket.makefile () as file_object:
    pickle.dump (..., file_object)
\end{lstlisting}
    \end{block}
    \begin{block}{JSON Stream ?}
\begin{lstlisting}[language=python,style=mini]
import json
with socket.makefile () as file_object:
    json.dump (..., file_object)
\end{lstlisting}
    \end{block}
    \begin{block}{YAML Stream ?}
\begin{lstlisting}[language=python,style=mini]
import yaml
with socket.makefile () as file_object:
    yaml.dump (..., file_object)
\end{lstlisting}
    \end{block}
\end{frame}


\begin{frame}[fragile]{Python Marshalling}
    \begin{block}{Byte Stream ?}
\begin{lstlisting}[language=python,style=mini]
data = 1234
socket.send (data.to_bytes (4, 'little'))
\end{lstlisting}
    \end{block}
    \begin{block}{Byte Stream ?}
\begin{lstlisting}[language=python,style=mini]
from struct import *
data = pack ('bhiq', 1, 2, 3, 4)
socket.send (data)
\end{lstlisting}
    \end{block}
\end{frame}


\begin{frame}{Code Now ...}
    \commitstrip{http://www.commitstrip.com/en/2016/06/07/good-code}{good-code}{0.6}
\end{frame}


\section{Assignment Part II}


\begin{frame}{Assignment}
    \begin{block}{Languages}
        Implement Part I in at least two programming languages.
    \end{block}

    \bigskip

    \begin{block}{Interoperability}
        Make sure your clients and servers in both languages \\ are interchangeable:
        \begin{itemize}
            \item Run any client with any server.
            \item Basic fields are enough (YYYY-MM-DD HH:MM:SS).
            \item Use sensible defaults for other fields (TZ, DOW, DOY).
        \end{itemize}
    \end{block}
\end{frame}


\begin{frame}{Submission}
    \begin{block}{GitLab}
        Use your personal GitLab repository under \\
        \url{https://gitlab.mff.cuni.cz/teaching/nswi163/2020}.
    \end{block}
    \begin{block}{Requirements}
        \begin{itemize}
            \item Use assignment subdirectory.
            \item Include build scripts and README with instructions.
            \item Do not commit binaries or temporary build artifacts.
        \end{itemize}
    \end{block}
\end{frame}
