\part{Protocol Buffers: Marshalling}


\section{Technology Overview}


\begin{frame}{Technology Overview}
    \begin{block}{Goals}
        Provide platform independent structured data serialization framework.
    \end{block}

    \bigskip

    \begin{block}{Features}
        \begin{itemize}
            \item Platform independent data description language.
            \item Serialization code generation for multiple languages \\
                  (C++, Java, Python, Go, Ruby, JavaScript, Objective C, C\# ...).
            \item Binary transport format with compact data representation.
            \item Textual transport using JSON.
        \end{itemize}
    \end{block}

    \bigskip

    \hfill ... \url{http://developers.google.com/protocol-buffers}
\end{frame}


\begin{frame}[fragile]{Installation}
    \begin{block}{Fedora Packages}
\begin{lstlisting}[language=bash,style=mini]
> dnf install protobuf-devel
> dnf install python3-protobuf
\end{lstlisting}
    \end{block}

    \bigskip

    \begin{block}{Source Distribution}
\begin{lstlisting}[language=bash,style=mini]
> git clone -b 3.6.x http://github.com/protocolbuffers/protobuf
> cd protobuf
> ./autogen.sh
> ./configure --prefix=${HOME}/.local
> make -j 8
> make install
> export PATH=${HOME}/.local/bin:${PATH}
> export LD_LIBRARY_PATH=${HOME}/.local/lib:${LD_LIBRARY_PATH}
> export PKG_CONFIG_PATH=${HOME}/.local/lib/pkgconfig:${PKG_CONFIG_PATH}
\end{lstlisting}
    \end{block}

    \bigskip

    \hfill ... use supplied install-protobuf.sh script
\end{frame}


% Installation timing
% - 2018 version 3.6 in lab ~4 minutes


\begin{frame}[fragile]{Message Specification Example}
\begin{lstlisting}[style=mini]
syntax = "proto3";

package example;

message AnExampleMessage {
    uint32 some_integer = 1;
    sint32 another_integer = 2;
    string some_string = 8;
    repeated string some_more_strings = 11;
}

message MoreExampleMessages {
    repeated AnExampleMessage messages = 1;
}
\end{lstlisting}
\end{frame}


% Selected Version History
% - 3.6
%   - Dropped support for certain older programming language versions
% - 3.5
%   - Unknown fields preserved by default and can be discarded explicitly
% - 3.0
%   - Unknown fields discarded
%   - Required fields discarded


\section{Assignment Part I}


\begin{frame}{Assignment}
    \begin{block}{Server}
        Implement a server that will provide information on current time.
        \begin{itemize}
            \item The server should accept a spec of what fields to return.
            \item Fields should be standard YYYY-MM-DD HH:MM:SS.
        \end{itemize}
    \end{block}

    \begin{block}{Client}
        Implement a client that will query server time:
        \begin{itemize}
            \item Pick a random combination of fields.
            \item Query information on current time.
            \item Print the time.
        \end{itemize}
    \end{block}

    \begin{block}{Interoperability}
        Implement compatible clients and servers in two languages.
    \end{block}
\end{frame}


\begin{frame}[fragile]{Examples To Begin With ...}
\begin{lstlisting}[language=bash,style=mini]
> git clone http://github.com/D-iii-S/teaching-introduction-middleware.git
\end{lstlisting}
    \begin{block}{C}
\begin{lstlisting}[language=bash,style=mini]
> cd teaching-introduction-middleware/src/protocol-buffers-basic-usage/c
> cat README.md
\end{lstlisting}
    \end{block}
    \begin{block}{Java}
\begin{lstlisting}[language=bash,style=mini]
> cd teaching-introduction-middleware/src/protocol-buffers-basic-usage/java
> cat README.md
\end{lstlisting}
    \end{block}
    \begin{block}{Python}
\begin{lstlisting}[language=bash,style=mini]
> cd teaching-introduction-middleware/src/protocol-buffers-basic-usage/python
> cat README.md
\end{lstlisting}
    \end{block}
\end{frame}


\section{Message Encoding}


\begin{frame}{Message Encoding}
    \begin{block}{Goals}
        Compact structure with support for field removal and addition.
    \end{block}

    \bigskip

    \begin{block}{Features}
        \begin{itemize}
            \item Sequence of field key value pairs.
            \item Key is field index and type indication.
                \begin{itemize}
                    \item One of variable integer, explicit length, fixed length.
                    \item Not enough to tell the exact field type !
                \end{itemize}
            \item Primitive repeated fields packed.
            \item Total length not included !
        \end{itemize}
    \end{block}
\end{frame}


\begin{frame}{Variable Length Encoding}
    \begin{block}{Goals}
        Support integers clustered around zero more efficiently.
    \end{block}

    \bigskip

    \begin{block}{Features}
        \begin{itemize}
            \item Integer stored as variable number of 7 bit values.
            \item High bit set to zero for last byte.
            \item Little endian byte order.
            \item Signed variant.
        \end{itemize}
    \end{block}
\end{frame}


\section{Message Specification}


\begin{frame}[fragile]{Primitive Field Types}
    \begin{block}{Integer Types}
        \begin{description}[123456789012345]
            \item[(s)fixed(32|64)] Integers with fixed length encoding.
            \item[(u)int(32|64)] Integers with variable length encoding.
            \item[sint(32|64)] Integers with sign optimized variable length encoding.
        \end{description}
    \end{block}
    \begin{block}{Floating Point Types}
        \begin{description}[123456789012345]
            \item[float] IEEE 754 32 bit float.
            \item[double] IEEE 754 64 bit float.
        \end{description}
    \end{block}
    \begin{block}{Additional Primitive Types}
        \begin{description}[123456789012345]
            \item[bool] Boolean.
            \item[bytes] Arbitrary sequence of bytes.
            \item[string] Arbitrary sequence of UTF-8 characters.
        \end{description}
    \end{block}
\end{frame}


\begin{frame}[fragile]{More Field Types}
    \begin{block}{Oneof Type}
\begin{lstlisting}[style=mini]
message AnExampleMessage {
    oneof some_oneof_field {
        int32 some_integer = 1;
        string some_string = 2;
    }
}
\end{lstlisting}
    \end{block}
    \begin{block}{Enum Type}
\begin{lstlisting}[style=mini]
enum AnEnum {
    INITIAL = 0;
    RED = 1;
    BLUE = 2;
    GREEN = 3;
    WHATEVER = 8;
}
\end{lstlisting}
    \end{block}
\end{frame}


\begin{frame}[fragile]{More Field Types}
    \begin{block}{Any Type}
\begin{lstlisting}[style=mini]
import "google/protobuf/any.proto";
message AnExampleMessage {
    repeated google.protobuf.Any whatever = 8;
}
\end{lstlisting}
    \end{block}

    \bigskip

    \begin{block}{Map Type}
\begin{lstlisting}[style=mini]
message AnExampleMessage {
    map<int32, string> keywords = 8;
}
\end{lstlisting}
    \end{block}
\end{frame}


\section{Message Manipulation}


\begin{frame}[fragile]{C++ Message Basics}
    \begin{block}{Construction}
\begin{lstlisting}[language=c,style=mini]
AnExampleMessage message;
AnExampleMessage message (another_message);
message.CopyFrom (another_message);
\end{lstlisting}
    \end{block}
    \begin{block}{Singular Fields}
\begin{lstlisting}[language=c,style=mini]
cout << message.some_integer ();
message.set_some_integer (1234);
\end{lstlisting}
    \end{block}
    \begin{block}{Repeated Fields}
\begin{lstlisting}[language=c,style=mini]
int size = messages.messages_size ();
const AnExampleMessage &message = messages.messages (1234);
AnExampleMessage *message = messages.mutable_messages (1234);
AnExampleMessage *message = messages.add_messages ();
\end{lstlisting}
    \end{block}
\end{frame}


\begin{frame}[fragile]{C++ Message Serialization}
    \begin{block}{Byte Array}
\begin{lstlisting}[language=c,style=mini]
char buffer [BUFFER_SIZE];
message.SerializeToArray (buffer, sizeof (buffer));
message.ParseFromArray (buffer, sizeof (buffer));
\end{lstlisting}
    \end{block}

    \bigskip

    \begin{block}{Standard Stream}
\begin{lstlisting}[language=c,style=mini]
message.SerializeToOstream (&stream);
message.ParseFromIstream (&stream);
\end{lstlisting}
    \end{block}
\end{frame}


\begin{frame}[fragile]{Java Message Basics}
    \begin{block}{Construction}
\begin{lstlisting}[language=java,style=mini]
AnExampleMessage.Builder messageBuilder;
messageBuilder = AnExampleMessage.newBuilder ();
messageBuilder = AnExampleMessage.newBuilder (another_message);
AnExampleMessage message = messageBulder.build ();
\end{lstlisting}
    \end{block}
    \begin{block}{Singular Fields}
\begin{lstlisting}[language=java,style=mini]
System.out.println (message.getSomeInteger ());
messageBuilder.setSomeInteger (1234);
\end{lstlisting}
    \end{block}
    \begin{block}{Repeated Fields}
\begin{lstlisting}[language=java,style=mini]
int size = messages.getMessagesCount ();
AnExampleMessage message = messages.getMessages (1234);
List<AnExampleMessage> messageList = messages.getMessagesList ();
messagesBuilder.addMessages (messageBuilder);
messagesBuilder.addMessages (message);
\end{lstlisting}
    \end{block}
\end{frame}


\begin{frame}[fragile]{Java Message Serialization}
    \begin{block}{Byte Array}
\begin{lstlisting}[language=java,style=mini]
byte [] buffer = message.toByteArray ();
try {
    AnExampleMessage message = AnExampleMessage.parseFrom (buffer);
} catch (InvalidProtocolBufferException e) {
    System.out.println (e);
}
\end{lstlisting}
    \end{block}

    \bigskip

    \begin{block}{Standard Stream}
\begin{lstlisting}[language=java,style=mini]
message.writeTo (stream);
AnExampleMessage message = AnExampleMessage.parseFrom (stream);
\end{lstlisting}
    \end{block}
\end{frame}


\begin{frame}[fragile]{Python Message Basics}
    \begin{block}{Construction}
\begin{lstlisting}[language=python,style=mini]
message = AnExampleMessage ()
message.CopyFrom (another_message)
\end{lstlisting}
    \end{block}
    \begin{block}{Singular Fields}
\begin{lstlisting}[language=python,style=mini]
print (message.some_integer)
message.some_integer = 1234
\end{lstlisting}
    \end{block}
    \begin{block}{Repeated Fields}
\begin{lstlisting}[language=python,style=mini]
size = len (messages.messages)
message = messages.messages [1234]
message = messages.messages.add ()
\end{lstlisting}
    \end{block}
\end{frame}


\begin{frame}[fragile]{Python Message Serialization}
    \begin{block}{Byte Array}
\begin{lstlisting}[language=python,style=mini]
buffer = message.SerializeToString ()
message.ParseFromString (buffer)
message = AnExampleMessage.FromString (buffer)
\end{lstlisting}
    \end{block}

    \bigskip

    \begin{block}{Standard Stream}
\begin{lstlisting}[language=python,style=mini]
file.write (message.SerializeToString ())
message.ParseFromString (file.read ())
AnExampleMessage.FromString (file.read ())
\end{lstlisting}
    \end{block}
\end{frame}


\begin{frame}{Code Now ...}
    \commitstrip{http://www.commitstrip.com/en/2017/03/16/when-we-leave-coders-to-do-their-own-thing}{coders-own}{0.9}
\end{frame}


\section{Assignment Part II}


\begin{frame}{Assignment}
    \begin{block}{Performance}
        Measure the performance of your implementation.
    \end{block}

    \bigskip

    \begin{block}{Experiment Design}
        Stick to the following, or provide arguments for why not:
        \begin{itemize}
            \item Random field mix, each field with probability 1/2.
            \item Measure at least two minutes long traffic.
            \item Report average invocation throughput.
            \item No printing during measurement.
            \item Compare with past assignments.
        \end{itemize}
    \end{block}
\end{frame}


\begin{frame}[fragile]{Measuring Time}
    \begin{block}{C++}
\begin{lstlisting}[language=c,style=mini]
#include <time.h>
#include <stdint.h>
struct timespec time;
clock_gettime (CLOCK_MONOTONIC_RAW, &time);
uint64_t nanoseconds =
    (uint64_t) time.tv_sec * 1000000000 +
    (uint64_t) time.tv_nsec;
\end{lstlisting}
    \end{block}
    \begin{block}{Java}
\begin{lstlisting}[language=java,style=mini]
long nanoseconds = System.nanoTime ();
\end{lstlisting}
    \end{block}
    \begin{block}{Python}
\begin{lstlisting}[language=python,style=mini]
import time
nanoseconds = time.clock_gettime (time.CLOCK_MONOTONIC_RAW) * 1000000000
\end{lstlisting}
    \end{block}
\end{frame}
