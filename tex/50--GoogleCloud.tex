\part{Google Cloud: Secure Communication}


\section{Technology Overview}


\begin{frame}{RSA Refresher}
    \begin{block}{Public Key Cryptography}
        A key pair where data encrypted with one key (private or public) \\
        can be decrypted with the other one (public or private).
        \begin{itemize}
            \item Public key available, private key kept secret
            \item Encrypting with public key, signing with private key
        \end{itemize}
    \end{block}

    \bigskip

    $x^{(p-1)(q-1)} = 1 ~ (modulo ~ pq)$ \hfill ... for $p, q$ prime and \\ \hfill $x$ not commensurable with $pq$ \\
    pick $p$, $q$ \\
    have $n = pq$ and $\phi = (p-1)(q-1$) \\
    pick $e$, $d$ such that $ed = 1 ~ (modulo ~ \phi)$ \\
    then $(m^e)^d = m^{1+k(p-1)(q-1)} = m \cdot m^{k(p-1)(q-1)} = m ~ (all ~ modulo ~ n)$ \\

    \bigskip

    \hfill ... Martin Ouwehand: The (simple) Mathematics of RSA
\end{frame}


\begin{frame}{DH Refresher}
    \begin{block}{Shared Secret Agreement}
        A process through which parties can agree on a shared secret \\
        without actually transmitting the shared secret itself.
    \end{block}

    \bigskip

    have $p$ and $g$ where $g$ is a generator of multiplicative integer group modulo $p$ \\

    \bigskip

    Alice: pick $a$ and publish $g^a ~ (modulo ~ p)$ \\
    Bob: pick $b$ and publish $g^b ~ (modulo ~ p)$ \\

    then $(g^a)^b = (g^b)^a$ is a shared secret \\ 

\end{frame}


\begin{frame}{TLS Technology Overview}
    \begin{block}{Goals}
        Provide privacy and integrity guarantees in network communication.
    \end{block}

    \bigskip

    \begin{block}{Features}
        \begin{itemize}
            \item Ciper suite negotiation
            \begin{itemize}
                \item Key exchange (RSA, DHE, PSK ...)
                \item Encryption (AES GCM, AES CCM, AES CBC ...)
                \item Message authentication (MD5, SHA1, SHA256 ...)
            \end{itemize}
            \item Secure session key exchange
            \item Server authentication
            \item Data encryption
            \item Data integrity
        \end{itemize}
    \end{block}

    \hfill ... TLS 1.2 RFC 5246
\end{frame}


\begin{frame}[fragile]{TLS RSA Handshake Sketch}
\begin{lstlisting}[style=mini]
[CLT] Hello, I support these cipher suites,
      and here is my CLIENT RANDOM number

[SRV] Hello, I have picked cipher suite AES256-SHA256,
      here is my SIGNED SERVER CERTIFICATE
      and here is my SERVER RANDOM number

[CLT] Here is a random PRE MASTER SECRET encrypted with your RSA key

MASTER SECRET = function (PRE MASTER SECRET, CLIENT RANDOM, SERVER RANDOM)
various session keys = function (MASTER SECRET)

[CLT] Finished and here is encrypted hash of exchanged messages

[SRV] Finished and here is encrypted hash of exchanged messages
\end{lstlisting}
\end{frame}


\begin{frame}[fragile]{TLS DH Handshake Sketch}
\begin{lstlisting}[style=mini]
[CLT] Hello, I support these cipher suites,
      and here is my CLIENT RANDOM number

[SRV] Hello, I have picked cipher suite AES256-SHA256,
      here is my SIGNED SERVER CERTIFICATE
      and here is my SERVER RANDOM number

[SRV] Here is my signed SERVER DH PUBLIC KEY

[CLT] Here is my CLIENT DH PUBLIC KEY

PRE MASTER SECRET = function (CLIENT DH PUBLIC KEY, SERVER DH PUBLIC KEY)
MASTER SECRET = function (PRE MASTER SECRET, CLIENT RANDOM, SERVER RANDOM)
various session keys = function (MASTER SECRET)

[CLT] Finished and here is encrypted hash of exchanged messages

[SRV] Finished and here is encrypted hash of exchanged messages
\end{lstlisting}
\end{frame}


\section{Assignment Details}


\begin{frame}{Assignment}
    \begin{block}{Server}
        Implement a server that will provide information on current time.
        \begin{itemize}
            \item The server should accept a spec of what fields to return.
            \item Fields should be standard YYYY-MM-DD HH:MM:SS.
        \end{itemize}
    \end{block}

    \begin{block}{Client}
        Implement a client that will query server time:
        \begin{itemize}
            \item Pick a random combination of fields.
            \item Query information on current time.
            \item Print the time.
        \end{itemize}
    \end{block}

    \begin{block}{Security}
        The connection between the client and the server should be encrypted.
    \end{block}
\end{frame}


\begin{frame}[fragile]{Python Secure Connection Basics}
    \begin{block}{Server}
\begin{lstlisting}[language=python,style=mini]
key_data = open ('server.key', 'rb').read ()
crt_data = open ('server.crt', 'rb').read ()
credentials = grpc.ssl_server_credentials ([( key_data, crt_data )])

server = grpc.server (...)
server.add_secure_port (SERVER_ADDR, credentials)
\end{lstlisting}
    \end{block}

    \bigskip

    \begin{block}{Client}
\begin{lstlisting}[language=python,style=mini]
crt_data = open ('server.crt', 'rb').read ()
credentials = grpc.ssl_channel_credentials (root_certificates = crt_data)
channel = grpc.secure_channel (SERVER_ADDR, credentials)

stub = AnExampleServiceStub (channel)
\end{lstlisting}
    \end{block}
\end{frame}


\begin{frame}{OAuth Technology Overview}
    \begin{block}{Goals}
        Standard protocol for granting third party applications \\ limited access to HTTP accessible resources.
    \end{block}

    \bigskip

    \begin{block}{Features}
        \begin{itemize}
            \item Considers multiple client types
            \begin{itemize}
                \item Applications running in browser
                \item Server hosted applications acting on own behalf
                \item Server hosted applications acting on user behalf
            \end{itemize}
            \item Heavily uses browser request redirection
            \item Requires (mostly) encrypted communication
            \item Authentication represented by (secret) access token
        \end{itemize}
    \end{block}

    \bigskip

    \hfill ... OAuth 2.0 RFC 6749
\end{frame}


\begin{frame}{Authorization Process Participants}
    \begin{block}{Resource Owner}
        This is the end user who authorizes third party clients to access resources. \\
        The resource owner accesses the third party client through a browser.
    \end{block}
    \begin{block}{Resource Server}
        This is the server that provides access to resources \\
        when shown authorization in the form of access token.
    \end{block}
    \begin{block}{Third Party Client}
        This is the application that needs to access \\
        resources on behalf of resource owner.
    \end{block}
    \begin{block}{Authorization Server}
        This is the server that can authenticate the resource owner \\
        and issues access tokens as directed by the resource owner.
    \end{block}
\end{frame}


\begin{frame}[fragile]{Authorization Process Sketch}
\begin{lstlisting}[style=mini]
[OWN] Accesses an application link that needs authorization.

[APP] Responds with REDIRECT sending the browser to authorization server.
      The link includes CLIENT ID and SCOPE and arbitrary STATE.

[OWN] The browser follows the link to the authorization server.

[AUT] The server authenticates the user behind the browser.
      The user is then asked to grant authorization for SCOPE.
      The server concludes with REDIRECT back to the application.
      The link includes AUTHORIZATION CODE and associated application STATE.

[OWN] The browser follows the link to the application.

[APP] The application gets the AUTHORIZATION CODE from the link.
      The application asks the authorization server to convert
      the AUTHORIZATION CODE into an ACCESS TOKEN.

[AUT] The server generates the ACCESS TOKEN as requested.

[APP] The application accesses the resource server
      with the ACCESS TOKEN included in request header.
\end{lstlisting}
\end{frame}


% Server Applications
% - Process assumes application can keep access token secret
% - Steps
%   - Have user follow link to authorization request page
%     The link contains
%     - client ID (assigned previously by administration)
%     - redirect URI (registered previously with administration)
%     - expected response type (server applications ask for authorization code)
%     - scope (what authorization the server wants, application specific)
%     - state (arbitrary string, application specific)
%   - The authorization server will authenticate the user and ask for permission
%     If permission is granted then the authorization request page will
%     go to redirect URI and provide state and authorization code
%     and possibly scope if that changed
%   - The server application asks the authorization server
%     to exchange the authorization code for
%     an access token using a POST request
%     - sends redirect URI and code and client ID
%     - also sends client secret (secret, assigned previously by administration)
%     - gets back access token with expiration time and possibly refresh token
% - Notable
%   - The access token never passes through the browser

% Browser Applications
% - Run in a browser so cannot have client secret in code
% - Same as previous except last step
%   - Client secret is not sent
%   - Not clear why this is fine
%   - This is not the spec way
%     - Spec provides simplified implicit grant handshake
%     - Spec points out the client is not really authenticated

% Finally
% - The access token is simply inserted to HTTP request headers
%   - This is called bearer token
%   - Other types also possible

% Tidbits
% - Standard actually allows significant variability
% - TLS connections are required (almost) everywhere
% - If somebody steals access token they can use it elsewhere
% - The authorization server can also issue refresh token, which
%   can be used to obtain another access token once the current one expires


\begin{frame}{Google Cloud Platform Technology Overview}
    \begin{block}{Goals}
        Computing platform build on Google infrastructure resources and services.
    \end{block}

    \bigskip

    \begin{block}{Features}
        \begin{itemize}
            \item Tons of services
            \begin{itemize}
                \item Compute services (IaaS and PaaS and FaaS)
                \item Storage services (SQL, tables, documents, raw block storage)
                \item Networking (private networks, load balancing, content delivery)
                \item Big data processing
                \item Machine learning
                \item Management
            \end{itemize}
            \item Accessible through public interfaces
            \item Libraries for multiple languages
        \end{itemize}
    \end{block}

    \bigskip

    \hfill ... \url{http://cloud.google.com}
\end{frame}


\begin{frame}[fragile]{Installation}
    \begin{block}{Browser}
        \begin{itemize}
            \item Register for free trial at \url{http://cloud.google.com}
            \item Log in to console at \url{http://console.cloud.google.com}
            \item Create a new project
            \item Enable required libraries
            \item Create and download a service account key
        \end{itemize}
    \end{block}
    \begin{block}{Shell}
\begin{lstlisting}[language=bash,style=mini]
> export GOOGLE_APPLICATION_CREDENTIALS=/path/to/service-account-key.json
\end{lstlisting}
    \end{block}
\end{frame}


\begin{frame}[fragile]{Cloud Speech API}
\begin{lstlisting}[language=python,style=mini]
from google.cloud import speech as google_cloud_speech
from google.cloud.speech import enums as google_cloud_speech_enums
from google.cloud.speech import types as google_cloud_speech_types

client = google_cloud_speech.SpeechClient ()

content = read_data_from_file (...)
audio = google_cloud_speech_types.RecognitionAudio (content = content)
config = google_cloud_speech_types.RecognitionConfig (language_code = 'en-US')

result = client.recognize (config, audio)
\end{lstlisting}
    \bigskip
    \hfill ... \url{http://cloud.google.com/speech/docs}
\end{frame}


\begin{frame}[fragile]{Cloud Translate API}
\begin{lstlisting}[language=python,style=mini]
from google.cloud import translate as google_cloud_translate

client = google_cloud_translate.Client ()

# Get a list of all supported languages.
languages = client.get_languages ()

# Translate a sentence.
result = client.translate ('some text', target_language = 'en')
\end{lstlisting}
    \bigskip
    \hfill ... \url{http://cloud.google.com/translate/docs}
\end{frame}


\begin{frame}{Assignment}
    \begin{block}{Goal}
        Create a client that translates input speech.
        \begin{itemize}
            \item An audio file with speech in English on input
            \item A text with speech translated into Czech on output
        \end{itemize}
    \end{block}

    \bigskip

    \begin{block}{Implementation}
        Use the client libraries rather than generated stub code.
    \end{block}
\end{frame}
