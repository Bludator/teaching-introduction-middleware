\part{Swagger: REST API Generation}


\section{Technology Overview}


\begin{frame}{REST: Representational State Transfer}
    \begin{block}{Features}
        REST compliant web services allow requesting systems to access
        and manipulate textual representations of web resources using
        a uniform and predefined set of stateless operations.

        \hfill ... Wikipedia
    \end{block}

    \bigskip

    Practically: each object (for example each database record) has its own URL
    and each action on the object a specific method or a specific child URL.

    \medskip

    \begin{itemize}
        \item Add new person with POST at \url{http://example.com/person/add}
        \item Get person info with GET at \url{http://example.com/person/42}
        \item Update person info with POST at \url{http://example.com/person/42}
        \item Delete person info with DELETE at \url{http://example.com/person/42}
    \end{itemize}

\end{frame}


\begin{frame}{REST: Motivation}
    \begin{block}{Motivation}
        Strike balance between \\ need for \emph{explicit interfaces} \\ and need for \emph{loose coupling}.
    \end{block}

    \medskip

    \begin{itemize}
        \item Standard communication protocol (HTTP)
        \begin{itemize}
            \item Already defines CRUD operations
            \item Provides security and reliability
            \item Is easy to deploy across internet
        \end{itemize}
        \item Encourages separating model from view
        \item Supports independent implementation technology \\ between client and server
    \end{itemize}
\end{frame}


\begin{frame}{REST and CRUD}
    \begin{block}{CRUD}
        \begin{description}
             \item[Create] to create an object
             \item[Read] to query object attributes
             \item[Update] to update object attributes
             \item[Delete] to delete an object
        \end{description}
    \end{block}

    \medskip

    \begin{itemize}
        \item The recommended minimum set of operations
        \item Corresponds reasonably well to HTTP methods
        \item Anything beyond CRUD is not considered pure REST
    \end{itemize}
\end{frame}


\begin{frame}[fragile]{REST: Data Transfer}

    Data exchange format is application specific but there are obvious choices
    \begin{itemize}
        \item JSON because of JavaScript in the browser
        \item XML because of existing library support
    \end{itemize}

\begin{lstlisting}[language=json,style=mini]
{
    "name": "Jane Doe",
    "email":  "jane.doe@example.com",
    "url": [
        "http://example.com/~jane.doe",
        "http://example.com/people/jane.doe"
    ],
    "address": {
        "street1": "Our Street One",
        "street2": "Street Line Two",
        "city": "The City",
        "postal": "12345"
    },
    "room": 123
}
\end{lstlisting}
\end{frame}


\begin{frame}{Swagger: API Development for REST}
    \begin{block}{Interface Description}
        \begin{description}
            \item[URLs] to identify data model classes
            \item[Actions] to operate on class instances
            \item[Attributes] with types to describe class instances
            \item[Security] defines access rules
            \item[Comments] provide human readable description
        \end{description}
    \end{block}

    \medskip

    \begin{itemize}
        \item Code generation
        \begin{itemize}
            \item Stubs wrap communication in language or framework specific constructs
            \item RPC style with futures for client
            \item Callback style for server
            \item Over 80 targets supported
        \end{itemize}
        \item Editor at \url{http://editor.swagger.io}.
    \end{itemize}
\end{frame}


\section{Assignment Details}


\begin{frame}{Assignment}
    \begin{block}{Inventory Application}
        Keeps track of \emph{users} and \emph{assets}. \\
        Basic user related operations are already defined. \\
        Define similar operations for assets and implement everything.
    \end{block}

    \begin{itemize}
        \item Interface
        \begin{itemize}
            \item Elementary CRUD operations for assets
            \item One to many relationship between users and assets
        \end{itemize}
        \item Server
        \begin{itemize}
            \item Python implementation using Flask, or
            \item Java implementation using Spring
        \end{itemize}
        \item Client
        \begin{itemize}
            \item TypeScript implementation using Angular, or
            \item R and bash helper scripts
        \end{itemize}
    \end{itemize}
\end{frame}


\begin{frame}[fragile]{Assignment Interface: Prologue}
\begin{lstlisting}[language=yaml,style=mini]
swagger: 2.0

info:
  description: Inventory database service
  version: 1.0.0
  title: Inventory
  termsOfService: ""
  license:
    name: Apache 2.0
    url: "http://www.apache.org/licenses/LICENSE-2.0.html"

host: localhost:8080

basePath: /v1

schemes:
  - http
\end{lstlisting}
\end{frame}


\begin{frame}[fragile]{Assignment Interface: Listing Users}
\begin{lstlisting}[language=yaml,style=mini]
paths:
  /users:
    get:
      operationId: readUsers
      produces:
        - "application/json"
      responses:
        200:
          schema:
            type: array
            items:
              $ref: "#/definitions/UserBase"

definitions:
  UserBase:
    type: object
    properties:
      id:
        type: integer
      firstname:
        type: string
      lastname:
        type: string
\end{lstlisting}
\end{frame}


\begin{frame}[fragile]{Assignment Interface: Querying User Data}
\begin{lstlisting}[language=yaml,style=mini]
  /user/{id}:
    get:
      summary: Query user information.
      operationId: readUser
      parameters:
        - in: path
          name: id
          description: ID of the user.
          required: true
          type: integer
      produces:
        - "application/json"
      responses:
        200:
          description: Successful operation
          schema:
            type: object
            $ref: "#/definitions/User"
\end{lstlisting}
\end{frame}


\begin{frame}[fragile]{Assignment Interface: Updating User Data}
\begin{lstlisting}[language=yaml,style=mini]
    post:
      summary: Update user information.
      operationId: updateUser
      consumes:
        - "application/json"
      produces:
        - "application/json"
      parameters:
        - in: path
          name: id
          description: ID of the user.
          required: true
          type: integer
        - in: body
          name: body
          description: Updated data.
          required: true
          schema:
            $ref: "#/definitions/User"
      responses:
        405:
          description: Invalid input
\end{lstlisting}
\end{frame}


\begin{frame}[fragile]{Assignment Interface: Inheritance}
\begin{lstlisting}[language=yaml,style=mini]
definitions:
  UserBase:
    type: object
    properties:
      id:
        type: integer
      firstname:
        type: string
      lastname:
        type: string
      email:
        type: string
  User:
    allOf:
      - $ref: "#/definitions/UserBase"
      - type: object
        properties:
          homepage:
            type: string
          department:
            type: string
\end{lstlisting}
\end{frame}


\begin{frame}[fragile]{Code Generation}
\begin{lstlisting}[language=bash,style=mini]
> swagger-codegen generate -i api.yaml -o <path> -l <framework>
\end{lstlisting}

    \begin{block}{Assignment}
        Use scripts \lstinline|build-{server,client}-*.sh| \\
        after updating \lstinline{api.yaml} to invoke code generator.
    \end{block}
\end{frame}


\begin{frame}{Flask-Based and Spring-Based Servers}
    \begin{block}{General}
        \begin{itemize}
            \item No real database (data kept in memory)
            \item Data dump to JSON at termination for debugging
            \item See \lstinline{README} for instructions how to run
        \end{itemize}
    \end{block}
\end{frame}


\begin{frame}[fragile]{Flask-Based Server}
\begin{block}{\lstinline{swagger\_server/controllers/default\_controller.py}}
\begin{lstlisting}[language=python,style=mini]
def create_user(body):  # noqa: E501
    """Creates a new user.

    :param body: User to be added.
    :type body: dict | bytes

    :rtype: None
    """
    if connexion.request.is_json:
        body = User.from_dict(connexion.request.get_json())
    return 'do some magic!'
\end{lstlisting}
\end{block}

    \begin{block}{\lstinline{controllers/users.py}}
        Actual implementation with data kept in memory.
    \end{block}
\end{frame}


\begin{frame}[fragile]{Spring-Based Server}
    \begin{block}{\lstinline{src/gen/java/io/swagger/api/UsersApiController.java}}
\begin{lstlisting}[language=java,style=mini]
public ResponseEntity<Void> createUser (
    @ApiParam (value = "User to be added." ,required=true)
    @Valid
    @RequestBody
    User body)
{
    String accept = request.getHeader("Accept");
    return new ResponseEntity<Void> (HttpStatus.NOT_IMPLEMENTED);
}
\end{lstlisting}
    \end{block}

    \begin{block}{\lstinline{src/main/java/io/swagger/api/UsersApiController.java}}
        Actual implementation with data kept in memory.
    \end{block}
\end{frame}


\begin{frame}{Angular-Based Client}
    \begin{block}{Goal}
        Add interface components for listing complete inventory. \\ Extend user detail page with asset list.
    \end{block}

    \begin{block}{General}
        \begin{itemize}
            \item Sources are under \lstinline{src/app}
            \item \lstinline{*.component.html} contains web page snippets of the component
            \item \lstinline{*.component.ts} contains TypeScript implementation of the component
        \end{itemize}
    \end{block}
\end{frame}


\begin{frame}{Angular-Based Client}
    \begin{block}{\lstinline{app-routing.module.ts}}
        \begin{itemize}
            \item Import all your components
            \item Add new routes to \lstinline{routes}
        \end{itemize}
    \end{block}

    \begin{block}{\lstinline{app.component.html}}
        \begin{itemize}
            \item Items in the topbar
        \end{itemize}
    \end{block}
\end{frame}


\begin{frame}[fragile]{Angular-Based Client: Reading Server Data}
    \begin{block}{\lstinline{users/users.component.ts}}
\begin{lstlisting}[language=typescript,style=mini]
export class UsersComponent implements OnInit {
    users: User [];

    constructor (private api: DefaultService) {}

    ngOnInit () {
        this.api.readUsers ().subscribe (u => this.users = u);
    }
}
\end{lstlisting}
    \end{block}

    \begin{block}{\lstinline{users/users.component.html}}
\begin{lstlisting}[language=html,style=mini]
<ul>
    <li *ngFor="let user of users">
        <a routerLink="/user/{{user.id}}">{{user.lastname}}, {{user.firstname}}</a>
    </li>
</ul>
\end{lstlisting}
    \end{block}
\end{frame}


\begin{frame}[fragile]{Angular-Based Application: Writing Server Data}
    \begin{block}{\lstinline{users/user.component.html}}
\begin{lstlisting}[language=html,style=mini]
<form (ngSubmit)="save();">
    <label for="user-first-name">First name:</label>
    <input [(ngModel)]="user.firstname" id="user-first-name" />
    ...
    <button type="submit">Save</button>
</form>
\end{lstlisting}
    \end{block}

    \begin{block}{\lstinline{users/user.component.ts}}
\begin{lstlisting}[language=typescript,style=mini]
export class UserComponent {
    save (): void {
        const id = +this.route.snapshot.paramMap.get ('id');
        this.api.updateUser (id, this.user).subscribe ();
    }
}
\end{lstlisting}
    \end{block}
\end{frame}


\begin{frame}[fragile]{Bash Client: Overview}
    \begin{block}{Generated}
        The generated script \lstinline{client.sh} is a thin wrapper on top of \lstinline{curl} doing the actual requests. Useful to check that the server works as expected.
    \end{block}

    \begin{block}{\lstinline{make-check-lists} and \lstinline{add-employees.sh}}

        Downloads list of employees, creates printable version of the inventory.

        Reads employee list from a CSV, adds them to the database.
    \end{block}

    \begin{block}{Task}
        Extend the \lstinline{make-check-lists} to include assets listing and create a similar script for adding assets.

\begin{lstlisting}
asset,price,acquired,owner
Magic Wand,42,2017,harry.potter@example.com
...
\end{lstlisting}
    \end{block}
\end{frame}


\begin{frame}[fragile]{Bash Client: Usage}
\begin{lstlisting}[language=bash]
> ./client.sh --silent readUsers | json_reformat

> ./client.sh --silent readUser id=1

> ./client.sh createUser \
    firstname==Horatio lastname==Hornblower \
    email==horatio.hornblower@royalnavy.mod.uk \
    department==Navy \
    homepage==https://www.royalnavy.mod.uk/hornblower
\end{lstlisting}
\end{frame}


\begin{frame}[fragile]{R Client: Overview}
    \begin{block}{\lstinline{dept-plot.r}}
        Draws a barplot showing number of employees in each department.
    \end{block}

    \begin{block}{Task}
        Create a similar script that will show total price of
        assets across departments and for each employee.
    \end{block}
\end{frame}


\begin{frame}[fragile]{\lstinline{dept-plot.r}}
\begin{lstlisting}[language=r,style=mini]
source ("init.r")

api <- DefaultApi$new ()

all.users.id <- api$read_users ()$content$id

department.people.count <- list ()

for (i in all.users.id) {
    u <- api$read_user (i)$content
    dept <- u$department
    if (!(dept %in% names (department.people.count))) {
        department.people.count [[ dept ]] <- 0
    }

    department.people.count [[ dept ]] <- department.people.count [[ dept ]] + 1
}

barplot (unlist (department.people.count), main="Employee count per department")
\end{lstlisting}
\end{frame}


% http://www.commitstrip.com/en/2016/08/25/a-very-comprehensive-and-precise-spec


\begin{frame}{Assignment Summary}
    \begin{itemize}
        \item Extend \lstinline{api.yaml} with assets-related operations and data definitions
        \item Extend one of the servers (Flask or Spring)
            \begin{itemize}
                \item Implement all CRUD operations and listing (all and per-user)
            \end{itemize}
        \item Extend one of the clients (Angular or R and bash)
            \begin{itemize}
                \item Angular: allow all of CRUD operations on assets and per-user listing
                \item R and bash: asset adding script, printable version of asset listing and two plotting scripts
            \end{itemize}
    \end{itemize}
\end{frame}
